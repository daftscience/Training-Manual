% \prettyimage{width=\textwidth}{graphics/result_field_all_entered}
\section{Flags}
Any result which \textit{may} require special attention will be flagged. These flags appear as letters in the \textsc{Flags} Column of \gls{are}.\\

\noindent
\begin{table}
    \begin{tabular}{lll}
        \large{\boldcap{\large Type}} & \boldcap{\large Symb.} & \boldcap{\large Meaning}\\
        \hline
        \textsc{Review}    & R & The results needs to be reviewed before verifying.\\
        \textsc{Delta}     & D & The result is has changed significantly.\\
        \textsc{High}      & H & The result is above the Normal Range.\\
        \textsc{Low}       & L & The result is below the Normal Range.\\
        \textsc{Critical}  & C & The result is Critical.\\
        \textsc{Linearity} & N & The result is outside of the defined Linearity.\\
        \textsc{Notify}    & @ & The result needs to be called to the floor.\\
        \textsc{Interpretation}    & (t) & The result is an an interpretation.\\
        \textsc{Comment}   & \textit{f} & The result contains a Comment or Note. \\
        % \textsc{Image}   & \textit{f} & The result contains a Comment or Note. \\
        \hline
    \end{tabular}
    \caption{Result Flags}
    \label{table:flags}
\end{table}

\newthought{The example below} shows multiple flags\sidenote{Specifically:
\begin{itemize}
    \item{\textsc{High}}
    \item{\textsc{Critical}}
    \item{\textsc{Review}}
    \item{\textsc{Linearity}}
    \item{\textsc{Comment}}
\end{itemize}
} for the result.\\

\prettyimage{width=.9\textwidth}{graphics/all_flags}

\section{Review Range}
The \textsc{Review Range} is used to prevent results from auto-verifying, and to alert the user that action might need to be taken.

\section{Critical Results}

\gls{are} will emphasize critical results by changing the color to \textsc{Red} and adding the \textsc{Critical Flag (C)}.\\

\prettyimage{width=\textwidth, trim={0 48 50 48}, clip}{graphics/result_field_all_entered}

\section{Linearities}
The \textsc{Linearities} have been defined as the minimum and maximum reportable values \boldcap{after} dilution.

Any result that does not fall within the \textsc{Linear Range} will automatically be converted to its inequality,\sidenote{\eg{A glucose of 2,401mg/dL will convert to >2,400mg/dL.}} and the \textsc{Linearity Flag (N)}\sidenote{\boldcap{N} is used because \boldcap{L} was taken by the \boldcap{Low} flag.} will be added.

\infoblock{The best way to manually enter results which are out of linearity is to enter a result which is outside of the linearity.\sidenote{Protip brought to you by Cpt. Obvious.}

When the instrument sends over results which are outside of the \textsc{Linear Range}, it will automatically be converted to its inequality.}

\subsection{Linearities and Calculations}

When a component of a calculation falls outside of its \textsc{Linear Range}, Cerner will perform the calculation using the linear limit of the assay.\sidenote{\eg{If the \textsc{U Timed Protein is <6mg/dL, Cerner will perform the calculation using a value of 6mg/dL.}}}

Finally, the appropriate inequality will be attached to the calculated result.

\newthought{The following two examples} show what happens if a component is above, and below linearity.\\

\begin{description}

\bolditem{Below Linearity} \\

\prettyimage{width=.8\linewidth}{graphics/inequality}\marginnote{In this case, we can accurately say the U24 Protein is <196mg/day.}\\

\bolditem{Above Linearity}\\

\prettyimage{width=.8\textwidth}{graphics/inequality_upper}\marginnote{In this case, we can accurately say the U24 Protein is >48,975mg/day.}\\

\end{description}

\newthought{If multiple components} of the calculation are outside of \textsc{Linearity}, Cerner will not be able to perform the calculation. Instead, a message will appear stating it's unable to perform the calculation.\\

\prettyimage{width=.75\textwidth}{graphics/multiple_inequalities}

\newthought{In this situation} the calculated result needs to be manually entered as \textsc{Unable to Calculate.}\sidenote{For more information \refpt{sec:convert_result}.}

\paragraph{Convert} The result to an \textsc{Alpha Response.}

\paragraph{Select} \textsc{Unable to Calculate.}

\infoblock{If you're unable to convert it to an \textsc{Alpha Response}, convert it to a \textsc{Freetext} and manually type: \textit{``Unable to Calculate.''}}

\subsection{Linearities and Delta Checks}
Cerner will \boldcap{not} be able to perform \textsc{Delta Checking} if any of the two values are outside of linearity.

\section{Reference Ranges}
Results which are lower than the defined \textsc{Reference Range} will be flagged as \textsc{Low (L)}, results which are higher will flag as \textsc{High (H.)}\sidenote{What else can I say?}



