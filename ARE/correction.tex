\textsc{Correction Mode} is used to modify results, or result comments of assays which have already been \textsc{Verified}.

When results are corrected their flags will be re-evaluated, and calculations\sidenote{Some calculations are more complicated and may need to be corrected manually. Examples are \boldcap{Platelet Mapping, and lipids.}} which use this value will be recalculated.

\paragraph{Click} \boldcap{Mode} from the menu bar.

\paragraph{Click} \boldcap{Correction}.\\

\prettyimage{width=.5\textwidth}{graphics/correction_mode}

\newthought{The Correction Mode Icon} \btn{graphics/correction_mode_icon} will indicate that \gls{are} is in \textsc{Correction Mode}.\\

\prettyimage{width=.8\textwidth}{graphics/are_correction}

\section{Correcting Results}

\paragraph{Select} the result which needs to be modified.\\

\prettyimage{width=.8\textwidth}{graphics/selected_for_correction}

\paragraph{Enter} the new result.

\paragraph{Hit} {\faKeyboardO} \boldcap{Enter} on the keyboard.\\

\prettyimage{width=.8\textwidth}{graphics/corrected_result}

\infoblock{Notice that the \boldcap{AGAP} has been recalculated and the flags have been updated.}

\paragraph{Hit} {\faKeyboardO} \textsc{Ctrl+A} to add a call comment\sidenote{This is simply to document the call.} to the modified results.\sidenote{For more information: \checkref{part:comments}{\refpt{part:comments}}{Refer to the \boldcap{Comments} Documentation}.}

\warnblock{The Correction comment must be entered \textbf{before} proceeding with the next step. Otherwise, the result will need to be Corrected again to add the comment.}

\paragraph{Click} \btn{graphics/correct} when finished.\\

\prettyimage{width=.8\textwidth}{graphics/corrected}

\infoblock{Notice that the \textsc{Status} has been changed to \textsc{Corrected}.}


\newthought{Finally, leave Correction Mode}. This step is very important; it will cause frustration if you don't.

\paragraph{Click} \boldcap{Mode} from the menu bar.

\paragraph{Click} \boldcap{Correction}.\\

\prettyimage{width=.5\textwidth}{graphics/leave_correction_mode}

\newthought{The Correction Mode Icon} \btn{graphics/correction_mode_icon} should no longer be visible in the lower left corner of \gls{are}.

\importantblock{Don't forget this step. It will prevent you from entering new results.}

\subsection{Viewing Original Results}

When results are modified in Cerner, all of the original information is saved.

\paragraph{Open} the \gls{accn} in \gls{orv}.\sidenote{\checkref{sec:orv_find_accn}{\refptchapp{part:orv}{sec:orv_find_accn}{orv}}{Refer to the \boldcap{Order Result Viewer} Documentation.}}

\paragraph{Double-Click} on the order.\\

\prettyimage{width=\textwidth}{graphics/corrected_view}

\paragraph{Select} the \textsc{Corrected} result.

\paragraph{Click} \btn{graphics/history}\\

\prettyimage{width=\textwidth}{graphics/history_view}

\newthought{This window will} show all the modifications which have been made to the result.

\importantblock{The current result will appear at the bottom of this list. It is sorted from oldest to newest.}



