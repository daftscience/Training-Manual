\Gls{are} is the Cerner application used to enter results. It has several modes which allow for manual result entry, resulting from instruments, or performing differentials.

\gls{are} can also show previous results, reference ranges, linearities, and critical ranges.

\section{About this Procedure}


Since resulting is the same in most of the \gls{are} modes,\sidenote{Differential Mode is the one exception.} \textsc{\autoref{ch:resulting} \nameref{ch:resulting} \textit{pg.}\pageref{ch:resulting}} will cover resulting orders using \acrlong{are}.

The chapters for each mode will simply walk you through the differences.

\section{Accession Result Entry Modes}
Depending on which tests are going to be resulted, \gls{are} has different modes to make resulting easier.

\begin{description}
    \bolditem{Accession Mode} can be used to enter results manually, or verify results performed by instruments.
    \bolditem{Differential Mode} is used to perform \textsc{Differentials} using Cerner. Mostly, it will be used for \textsc{Body Fluids}.

    It can be used during \textsc{Wam Downtimes} to perform CBC differentials.

    \bolditem{Instrument Mode} is used to review results which cannot be \textsc{Auto-Verified} by the instruments.

    \bolditem{Correction Mode} is used to correct results which have already been \textsc{Verified.}
\end{description}



