\section{Too Many Digits}

When trying to modify a result, you may get an alert stating that ``Too many Digits were entered.''\\

\prettyimage{width=.65\textwidth}{graphics/too_many_digits}

\newthought{This often happens when} changing a result value. When you \textsc{Double Click} on a result cell the cursor is placed to the far right.\sidenote{In the example, it's to the right of \_\_\_\_5.0|}\\

\prettyimage{width=.65\textwidth}{graphics/cursor_location}

Cerner treats the leading underscores as zeros.

\newthought{The easiest fix} is to hit {\faKeyboardO} \boldcap{ Backspace} to clear the results.\\

\prettyimage{width=.65\textwidth}{graphics/cursor_location_fixed}

\section{Decimals Not Allowed}

The number of places after the decimal point is defined for each assay.

If no decimals have been defined, Cerner will prevent them from being entered.\\

\prettyimage{width=.65\textwidth}{graphics/decimals_warning}

\section{Unable to Enter Results}

There are two reasons why \gls{are} won't let you enter results.

\begin{itemize}
    \item{The sample isn't \textsc{In-lab}}
    \item{ARE is in Correction Mode.}
\end{itemize}

\subsection{The Sample Isn't In-Lab}

Results can only be entered if status is \textsc{Pending}. This means that the sample has been \textsc{Logged-in} to the proper location.\sidenote{For additional help with this issue: \checkref{ch:ci_troubleshoot}{\refptchapp{part:ci}{ch:ci_troubleshoot}{container_inquiry}}{Refer to the \boldcap{Container Inquiry: }\textsc{Using Container Inquiry for Troubleshooting} Documentation}.}

\begin{quote}
\newthought{In this example} the \textsc{Ser hCG Ql} cannot be resulted.\\

\prettyimage{width=.8\textwidth}{graphics/not_in_lab}

\begin{description}
    \bolditem{Lytes Panel} Has a status of \textsc{Pending}. This \boldcap{can} be resulted.
    \bolditem{Ser hCG Ql} Has a status of \textsc{Not In-Lab}. This \boldcap{cannot} be resulted.
\end{description}
\end{quote}


\newthought{To fix this} the sample must be \textsc{Logged-in}. If you currently have the container, this can be done from within \gls{are}.

\paragraph{Click} \btn{graphics/login_icon} from the \textsc{Tool Bar}.

\paragraph{Log-in} the sample.\sidenote{For more information: \checkref{ch:login_accn}{\refptchapp{part:login}{ch:login_accn}{specimen_login}}{Refer to the \boldcap{Specimen Log-in Procedure}}.}


\subsection{You're Still in Correction Mode}

If you can see the \btn{graphics/correction_mode_icon} in the bottom left corner of \gls{are}, then you're currently in \textsc{Correction Mode}. New results cannot be entered if \gls{are} is in \textsc{Correction Mode}.

\newthought{To Fix this} disable \textsc{Correction Mode}.

\paragraph{Click} \boldcap{Mode} from the menu bar.

\paragraph{Click} \boldcap{Correction}.\\

\prettyimage{width=.5\textwidth}{graphics/leave_correction_mode}

\newthought{The Correction Mode Icon} \btn{graphics/correction_mode_icon} should no longer be visible in the lower left corner of \gls{are}.