\newthought{When tests are ordered} which require creatinine results Cerner will decide if a creatinine is required. If a creatinine is required, Cerner will add the creatinine automatically. This document will describe the process and procedures to ensure that the rules trigger properly.

\bigskip

\section{Glomerular filtration rate}

\newthought{With the GFR} there are two situations we will run into
\begin{itemize}
\item No creatinine ordered before the GFR was ordered.
    \begin{description}
        \item Cerner will automatically add a creatinine to the GFR. Once the creatinine is            resulted, the GFR will be calculated.
    \end{description}
\item Order for Creatinine placed \textit{before} the GFR was ordered.
    \begin{description}
        \item The lab will have to cancel the GFR and add it to the same Accession number as the creatinine, using Accession Add-on in Department Order Entry. The GFR will immediately calculate. \marginnote[-2\baselineskip]{The GFR needs to be on the same accession number as the GFR.}
    \end{description}
\end{itemize}


\section{Creatinine Clearance}

\newthought{With the creatinine clearance} Cerner will be able to determine if a plasma creatinine is required. If the patient does not have an existing order for a plasma creatinine, Cerner will order it automatically. If an order \textit{does} exist for a plasma creatinine, Cerner will use that result to calculation the Creatinine Clearance.\marginnote[-4\baselineskip]{Unlike the GFR, the serum creatinine does not need to be on the same accession number for the calculation to work.}



