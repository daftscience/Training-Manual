
\gls{login} is the application used to update sample locations within Cerner. This application will most commonly be used to log-in samples which have been collected by the floors.

\Gls{login} is also used to log-in ``Transfer Lists'' sent by other facilities.

Cerner will not allow orders to be resulted until they have been \textsc{Logged-in} to the testing laboratory. In most cases, the instruments will not know which tests to perform.

\newthought{Open} \gls{login} by clicking the \appicon{specimen_login} icon from the \gls{ab}.\\

\prettyimage{width=.8\textwidth}{graphics/login.png}

\newthought{There are three} way's samples can be logged-in:
\begin{description}
	\bolditem{List} Log-in an entire list. Either a \textit{Collection List}, or a \textit{Transfer List}
	\bolditem{Patient} This is to log samples in using a Patient Identifier. In \textit{most} cases, this method is not recommended,\sidenote{However, it may be useful for sites where the Laboratory collects samples.} so it will not be discussed in this procedure.
	\bolditem{Accession} Log-in samples using the Accession Number.\sidenote{\ie{ Scanning the bar-code.}} \refpt{ch:login_accn}.
\end{description}
