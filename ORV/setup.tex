When opening \gls{orv} for the first time, it is a good idea to change the default value for the ``Find Person'' field.

While there are many options which can be modified, it's best to avoid changing them until you've become familiar with how \gls{orv} works. Modifying these settings can make it difficult to find orders.

\paragraph{Open } \gls{orv} by clicking \appicon{orv} icon from the \gls{ab}.

\paragraph{Click} \btn{graphics/cancel_button} to close the ``Find Orders'' window.

\newthought{In the main} \gls{orv} window:
\begin{marginfigure}
\prettyimage{width=.9\textwidth}{graphics/taskbar_options.png}
\end{marginfigure}
\paragraph{Click} \boldcap{View} from the menu bar.
\paragraph{Select} \boldcap{Options\ldots}

\newthought{This will open} the \textsc{Options} window. The two options we are going to adjust are the \textsc{Find Person} option and the \textsc{Date Range}.\sidenote[][2\baselineskip]{This option will speed up the time it takes to search for orders.}\marginnote[1\baselineskip]{\warning{Again, avoid those other options. If you want to filter, it can be done in the ``Find Orders'' window.}}\\

\noindent
\begin{tikzpicture}
\node [anchor=west] (mrn) at  (-2, 4) {\textit{S e l e c t} MRN};
\node [anchor=west] (date) at (-2, 3) {\textit{U n c h e c k} All Dates};
\node [anchor=west] (la) at   (-2, 1.5) {\textit{S e t  t o} 1};
\node [anchor=west] (lb) at   (-2, .5) {\textit{S e t  t o} 30};
\begin{scope}[xshift=1.5cm]
    \node[anchor=south west,inner sep=0] (image) at (0,0) {\prettyimage{width=0.55\textwidth, trim={0 135pt 0 0}, clip}{graphics/options}};
    \begin{scope}[x={(image.south east)},y={(image.north west)}]
        % \draw[amethyst,ultra thick,rounded corners] (0.01,0.73) rectangle (0.2,0.79);
        \draw [-stealth, line width=3pt, teal400] (mrn) to[out=0, in=-200] (0.08,0.58);
        \draw [-stealth, line width=3pt, deeporange400] (date) to[out=0, in=-250] (0.1,0.3);
        \draw [-stealth, line width=3pt, indigo400] (lb) to[out=0, in=-200] (0.35,0.1);
        \draw [-stealth, line width=3pt, cyan400] (la) to[out=0, in=-200] (0.35,0.16);
    \end{scope}
\end{scope}
\end{tikzpicture}%

\paragraph{Click} \btn{graphics/shiny_ok}
