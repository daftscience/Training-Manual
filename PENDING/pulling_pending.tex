\section{About the Options}

When \gls{pi} is first opened, the \textsc{Select Pending Procedures} window will appear. This window is used to select which assays will be displayed.\\

\prettyimage{width=.6\textwidth}{graphics/select_procedures}

\subsection{Test Site}
The \textsc{Test Site} is the \treeitem{instrument}{Instrument}, \treeitem{Bench}{Bench}, \treeitem{section}{Section} or \treeitem{subsection}{Subsection} you're looking for.\sidenote{Selecting a \treeitem{section}{Section} or \treeitem{section}{Subsection} will include everything under it. For instance, \textsc{DC Heme S} will include: CBCs, Tegs, ESRs, \etc}

\infoblock{For a detailed description of routing: \checkrouting.}

\subsection{Procedure}

The \textsc{Procedure} is used to limit the list to one test.\sidenote{This is an optional field.}

\subsection{Status}

The \textsc{Status} is used to refine the list by the status of the order.

\begin{table}
\begin{tabular}{ll}
    \boldcap{\large Status} & \boldcap{\large What will be displayed on the list} \\
    \hline
    \textsc{All pending}           &  All Pending Orders.  \\
    \textsc{Received only}         &  Orders with at least one Received Container. \\
    \textsc{In-Lab only}           &  Orders which are in the In-Lab Status. \\
    \textsc{Scheduled only only}   &  Only view order that haven't been Dispatched.\\
\hline
\end{tabular}
\caption{Pending Status Options}\label{table:pending_status_Options}
\end{table}

\section{Entering Options}

\paragraph{Select} the \boldcap{In-Lab Only} option.\\

\noindent%draw top
\begin{tikzpicture}
    \begin{scope}
        \node[anchor=south west,inner sep=0] (image) at (0,0) {\prettyimage{width=.75\textwidth}{graphics/select_procedures} };
        \begin{scope}[x={(image.south east)},y={(image.north west)}]
            \draw [-stealth, line width=3pt, deeppurple400] (.4, 1.05) to[out=270, in=0] (0.25, 0.2);
        \end{scope}
    \end{scope}
\end{tikzpicture}%

\paragraph{Enter} your hospital's abbreviation in the \boldcap{Test Site} field.

\begin{marginfigure}
    \scshape
    \begin{tabular}{ll}
        \boldcap{Site} & \boldcap{Abbreviation} \\
        \hline
        Brack                & BH   \\
        Dell's               & DC   \\
        Edgar B.             & SEBD \\
        Hays                 & SHC  \\
        Highland Lakes       & SHL  \\
        SMCA                 & SMC  \\
        Northwest            & SNW  \\
        Southwest            & SSW  \\
        Williamson           & SWC  \\
        \hline
    \end{tabular}
\end{marginfigure}

\infoblock{If you know a portion of the site name, you can enter it now. For instance: If you're looking for \treeitem{section}{SMC Heme S} you can type: `SMC Heme S', or even just `SMC Heme.'}

\paragraph{Click} the \btn{graphics/elips} button.\sidenote{\hotkey{Enter} also works}\\

\newthought{To refine the} list, continue typing the name of the site in the text box.

\importantblock{For a map of your hospitals routing: \checkrouting
}


\paragraph{Find} the \treeitem{section}{Section}, \treeitem{subsection}{Subsection}, \treeitem{instrument}{Instrument} or \treeitem{bench}{Bench}.\sidenote{\eg{\treeitem{section}{SMC Heme S}}}\\

\prettyimage{width=.5\textwidth}{graphics/test_site_lookup_found}

\paragraph{Click} \btn{graphics/ok}\\

\prettyimage{width=\textwidth}{graphics/pending}

\section{Refreshing the Pending List}
It is useful to periodically refresh \textsc{Pending Inquiry.} This will remove any orders that have been \textsc{Validated}, or \textsc{Canceled} and add new orders which have been \textsc{Logged-In},

\paragraph{Click} the \btn{graphics/refresh_button} \textsc{Refresh Button} from the \textsc{Tool bar.}\\

\noindent
\begin{tikzpicture}
\begin{scope}
    \node[anchor=south west,inner sep=0] (image) at (0,0) {\prettyimage{width=\textwidth}{graphics/toolbar}};
    \begin{scope}[x={(image.south east)},y={(image.north west)}]
        \draw[indigo400, rounded corners, line width=3] (0.04, 0.1)rectangle (0.13, 0.98);
    \end{scope}
\end{scope}
\end{tikzpicture}%

\newthought{This will re-open} the \textsc{Select Pending Procedures} window.\\

\prettyimage{width=.75\textwidth}{graphics/refresh_window}

\paragraph{Click} \btn{graphics/ok} \sidenote{\textsc{Optionally}: Update any information.}

\section{Sorting Pending Inquiry}

The Pending List can be sorted by any of the columns displayed.

\paragraph{Click} on the \boldcap{Column Header}.\sidenote{The current Sorting column will appear in \boldcap{Bold} text.}

\noindent%draw top
\begin{tikzpicture}
    \begin{scope}
        \node[anchor=south west,inner sep=0] (image) at (0,0) {\prettyimage{width=\textwidth}{graphics/default_sort} };
        \begin{scope}[x={(image.south east)},y={(image.north west)}]
            \draw [-stealth, line width=3pt, teal200] (.3, 2) to[out=270, in=90] (0.05,1);
            \draw [-stealth, line width=3pt, teal300] (.3, 2) to[out=270, in=90] (0.2,1);
            \draw [-stealth, line width=3pt, teal400] (.3, 2) to[out=270, in=90] (0.4,1);
            \draw [-stealth, line width=3pt, teal500] (.3, 2) to[out=270, in=90] (0.5,1);
        \end{scope}
    \end{scope}
\end{tikzpicture}%

\infoblock{Clicking a header multiple times will toggle the sort order.\sidenote{Sort \faSortAlphaAsc  or  \faSortAlphaDesc} }


\section{Printing the Pending Log}

The pending log can be printed in its entirety or a portion of the pending log can be printed.

Cerner will use the \textsc{Default Printer} for your computer.\sidenote{To view the installed printers:\\

\noindent
\textit{H i t } the { \faWindows} key on the keyboard.\\
\noindent
\textit{S e l e c t } \boldcap{Devices and Printers}.}


\subsection{Printing the Entire Pending Log}

\paragraph{Click} the \textsc{Print} icon \btn{graphics/print_icon} from the \textsc{Tool bar}.\sidenote{\hotkey{Ctrl+P} will also work.}\\

\prettyimage{width=.4\textwidth}{graphics/print_dialog}

\paragraph{Select} \boldcap{All}.


\subsection{Printing selected procedures:}

\paragraph{Select} the orders to print.\sidenote{This can be done by clicking and dragging down the list.}

\paragraph{Click} the \textsc{Print} icon \btn{graphics/print_icon} from the \textsc{Tool bar}.\sidenote{\hotkey{Ctrl+P} will also work.}

\paragraph{Select} \boldcap{Selected.}

\paragraph{Click} \btn{graphics/ok}
