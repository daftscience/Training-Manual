\setcounter{page}{1}

\chapter{Quick reference}

In order to find your information quicker, you can use this page to answer some of the questions.


\answer{Your test patient's MRN\label{your_mrn}}

\subsection{Department Order Entry}
\answer{Accession Number (Blood)\label{blood_accession}}
\answer{Accession Number (Bone Marrow)}

\subsection{Container Inquiry}
\answer{Which container is the PTT on?\label{ptt_container}}
\answer{Which \boldcap{Service Resource} is the \boldcap{PTT} routed to?\label{ptt_sr}}

\subsection{Transferring}
\answer{What is the number of the list your PTT is on?\label{ptt_list}}
\answer{What is the number of the list your CHABM is on?\label{bm_list}}

\subsection{Advanced ARE}
\answer{Accession Number (Synovial)}
\clearpage

\chapter{Introduction}
Use this handout to follow along with the class. Each section contains some instructions and questions to be answered out.

Refer to this handout as you practice your own. If you have any questions you can contact LIS, or email me at:{\color{teal900} { tjperry@seton.org}}

\infoblock{Too keep track of your progress, place a {\faCheck} in the {\activityicon} as you complete them.}



\section{Accession Numbers}

    \answermulti[{Year, Julian Day}]{How are the first 5 digits of an accession number defined?}

    \answer[A letter which differentiates containers.]{What is a Container ID?}


\chapter{Placing Orders}

\section{Department Order Entry}

\activity{Set the default Values for DOE}\sideref{part:doe}{sec:doe_custom}

\activity{Change the \boldcap{Patient Identifier field} to search for MRN}\sidenote{\hotkey{Ctrl+Tab}}

\subsection{Ordering Tests}
Here we will be ordering some tests on your patient.

\warnblock{Please read the instructions carefully!
    \begin{itemize}
        \item Do \boldcap{NOT} change the client field.
        \item Do \boldcap{NOT} submit orders until instructed to do so.
    \end{itemize}
}

\activity{Search for your Patient using the MRN}\sideref{part:doe}{patientLookup}\sidenote{\refQuestion{your_mrn}}

\activity{Search for a Electrolytes Metabolic Panel}

\infoblock{Set the following values:
    \begin{itemize}
        \item{{\faCheckSquareO} Collected}
        \item Set the \boldcap{Received Location} to SMC Login.
    \end{itemize}

}

\activity{Add the Electrolytes Panel to your \gls{spad}}\sidenote{\hotkey{\boldcap{Ctrl+A}}}

\infoblock{For the next activity, you don't need to change any of the options.They will default to those chosen with the Electrolytes Panel.}

\activity{Add the following tests:}


\begin{quote}
\begin{quote}
    \begin{itemize}
        \item hCG Qualitative
        \item TNSIBC
        \item Chromosome Analysis on Bone Marrow
    \end{itemize}
\end{quote}
\end{quote}

\activity{Submit the Order}\sidenote{\hotkey{\boldcap{Ctrl+O}}}

\answerFP{Accession Number (Blood)}
\answerFP{Accession Number (Bone Marrow)}

\importantblock{For this next activity, set the \boldcap{Collection Time} to be greater than the \boldcap{Received time}.}

\activity{Order a CDSF}\sidenote{This will generate an ERROR.}

\answer[Error]{What is the \boldcap{Submission Status?}}

\activity{View the Submission Error}\sideref{part:doe}{sec:submission_errors}

\activity{Fix the error, and re-submit the orders}

\answerFP{Accession Number (Synovial)}

\subsection{Accession Add-On}

\activity{Add a PTT to the blood sample ordered earlier.}

\answer[Yes]{Should you override?}

\section{Container Inquiry}

\activity{Using \boldcap{Container Inquiry} search for the blood sample.}\sidenote{\refQuestion{blood_accession}}

\answerFP{What container is your PTT on?}

\answer[SMC Login]{What is the \boldcap{In-Lab Location} of the PTT?}

\answerFP{Which \boldcap{Service Resource} is the PTT routed to?}
\answerkey{SMC ACL1}



\chapter{Entering Results}

\section{Accession Result Entry}

\subsection{Getting Started}

\activity{Set the Default Values}\sideref{part:are}{ch:are_setup}


\subsection{Entering Results}

\importantblock{Do not \boldcap or \boldcap{Verify} these until instructed.}

\activity{Open the Accession number for your Blood Sample using ARE.}\sidenote{\refQuestion{your_mrn}}

\activity{Set the \boldcap{Test Site} to SMC Remisol.}\sidenote{Click Retrieve if needed.}

\activity{Enter results for the Lytes using the following criteria:}

\begin{quote}
\begin{quote}
    \begin{itemize}
        \item \boldcap{Sodium:} 201
        \item \boldcap{Potassium:} 6
        \item \boldcap{Chloride:} 100
        \item \boldcap{CO2:} 20
        \item \boldcap{Everything Else:} You can pick.
    \end{itemize}
\end{quote}
\end{quote}

\answer[>80]{What is the value of the Anion Gap?}

\answermulti[Unable to calculate]{Can the anion gap be calculated if both the Sodium and Chloride are out of linearity?}

\activity{Verify the panel.}

\answermulti[ARE tells you there are critical results. It asks you to enter a comment.]{What Happened?}

\importantblock{Normally, we select ``Yes.''  However, in order to show you additional features, we'd like you select ``No.''}

\activity{Click ``No''}

\activity{Enter a \boldcap{Batch Comment}}\sidenote{\hotkey{\boldcap{Ctrl+B}}}

\activity{Using the \boldcap{CallRed} template, enter a critical comment}\sideref{part:comments}{ch:using_templates}

\activity{Click Verify}

\answermulti[The critical results have comments]{Why didn't Cerner warn you that there are critical results?}

\subsection{Correction Mode}

In this section, we will be modifying some results.\\

\activity{Set ARE to \boldcap{Correction Mode}}

\activity{Change the Sodium to 136}

\answermulti[It changed to 16]{What happened to the Anion Gap?}

\answer[No]{Will Cerner warn you that a ``Corrected Comment'' needs to be entered?}

\activity{Enter a Correction Comment}\sidenote{\hotkey{\boldcap{Ctrl+A}}}

\answer[By correcting the result again.]{If you forgot to enter a comment, how can it be added?}

\answer[Comments are considered part of the result.]{Why do you need to do that?}

\chapter{Viewing Orders}


\section{Order Result Viewer}

\activity{Change default values}\sideref{part:orv}{ch:orv_getstart}

\subsection{Searching for a Patient}

\activity{Pull up your patient's orders}\sidenote{\refQuestion{your_mrn}}

\activity{Double click on the Electrolytes Panel.}

\answer[The results window Popped up.]{What Happened?}

\activity{Select a \boldcap{Corrected Result}}\sidenote{Look at the \boldcap{Status} column.}

\activity{Click on \boldcap{History}}

\answer[>200]{What was the original Sodium Value?}

\subsection{Canceling Orders}

\activity{Cancel the SHCG}

\answermulti[Click the little Comment icon.]{How do you enter a Comment?}

\section{Pending Inquiry}

\activity{Modify the headings}\sideref{part:pi}{ch:customizing_pending}

\activity{Open a Pending using the service resource of the PTT}\sidenote{\refQuestion{ptt_sr}}

\answermulti[Not selecting the correct service resource]{What would cause the transfer to fail?}
\activity{Re-Route the PTT to Brakenridge.}

\activity{Transfer the PTT.}\sideref{part:pi}{ch:transfer_from_pending}


\chapter{Transfer Specimen}

\activity{Set up defaults}

\infoblock{In this section we will create two transfer lists. Please pay close attention, this section gets complicated.}


\subsection{Creating Lists}

\subsection{Your First List}

The first list will contain your PTT.

\answer[From SMC Login]{What will the \boldcap{From} location be?}

\answer[To BH]{What will the \boldcap{To} location be?}

\activity{Create a transfer list for the PTT Order.}\sideref{part:transfer}{ch:creat_trans_list}

\answerFP{What is your list number?}


\subsection{Your Second List}

The second list will contain your CHABM.

\answer[From SMC Login]{What will the \boldcap{From} location be?}

\answer[To BH Referrals]{What will the \boldcap{To} location be?}

\activity{Create a transfer list for the CHABM}\sideref{part:transfer}{ch:creat_trans_list}

\answerFP{What is your list number?}

\subsection{Modify List}

\activity{Select and Modify your First List}\sidenote{\refQuestion{ptt_list}}

\activity{Add the TNSIBC To the list}\sidenote{\refQuestion{blood_accession}}



\chapter{Specimen Login}

\activity{Open Accession Log-in and choose: \boldcap{Log-in by Accession} }\sideref{part:login}{sec:login_procedure}

\activity{Make Sure the \textsc{Log-in Location} is set to \boldcap{BH Login}}

\importantblock{In order to prevent errors, enter the container ID after the Accession Numbers.\sidenote{Capital Letters}}

\begin{itemize}
    \item CHABM\sidenote{This will be container `A'}
    \item PTT\sidenote{\refQuestion{ptt_container}}
\end{itemize}


\activity{Login By List}

\activity{Open Accession Log-in and choose: \boldcap{Log-in by List}}\sideref{part:login}{ch:login_list}

\activity{Log in your Second List}\sidenote{\refQuestion{bm_list}}

\answer[No orders on list.]{What does the message say?}

\activity{Log in your First List}

\answer[The TNSIBC Showed up]{What happened?}

\answer[Because it was Missed]{Why?}


\chapter{Advanced ARE}

\activity{Open your Synovial Fluid in ARE}

\activity{Result it so that a Differential and Crystal ID will reflex.}

\infoblock{\begin{itemize}
        \item \boldcap{WBC} > 5
        \item \boldcap{Crystals}: Present
    \end{itemize}
}

\section{Differential Mode}

\activity{Perform a Differential.}\sideref{part:are}{sec:perform_differential}

\answer[The Total Cells Counted]{What must you always remember?}

\activity{Switch to Accession Mode and enter the Total Cells Counted.}

